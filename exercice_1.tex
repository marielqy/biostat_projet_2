\documentclass[../main.tex]{subfiles}

\begin{document}
\begin{CJK*}{UTF8}{gbsn}
\section*{Exercice 1}
Décrivez brièvement les variables du jeu de données 
\texttt{diabetic} dans la bibliothèque \texttt{survival} en R.
Quelle était une question de recherche menant à cette collecte de données?
Produisez un graphique de Kaplan-Meier utilisant l'échantillon entier,
le groupe de \texttt{trt = 0} et le groupe de \texttt{trt = 1} 
pour le temps jusqu'à devenir aveugle.
Pour le groupe \texttt{trt = 0}, trouvez le temps médian de survie et
construisez un intervalle de confiance à $95\%$ pour le temps médian de survie.
Faites un test de Log-Rang pour comparer les deux groupes.

Finalement, on se demande si d'autres variables
dans le jeu de données pourraient être des facteurs confondants, et si l'on devrait stratifier le
test du Log-Rang sur l'une de ces variables. À partir de statistiques descriptives, de graphiques
et/ou d'arguments adaptés au contexte de l'étude, discuter de laquelle des variables du jeu de
données risque d'agir comme facteur confondant, et reproduire le test du Log-Rank stratifié pour cette variable.

\paragraph{Solution}
\section{a}
\begin{itemize}
  \item \textbf{ID} : Il est utilisé pour distinguer chaque participant dans l'ensemble de données.
  \item \textbf{laser} : Il s'agit du type de traitement laser reçu. 1=xenon, 2=argon
  \item \textbf{age} : Il s'agit de l'âge auquel le diabète a été diagnostiqué chez le patient.
  \item \textbf{eye} : Il s'agit d'un facteur avec des niveaux de gauche et de droit.
  \item \textbf{trt} : Il s'agit du groupe de traitement. 0=no treatment, 1=laser
  \item \textbf{risk} : Il s'agit d'une variable quantitative d'évaluation du risque utilisée pour classer les participants dans différents groupes de risque (les valeurs varient de 6 à 12).
  \item \textbf{time} : Il s'agit de la date de l'événement ou de la dernière visite de suivi.
  \item \textbf{status} : Il s'agit d'une variable binaire utilisée pour indiquer si une perte de vision s'est produite au cours de la période d'étude, où 0 indique qu'aucune perte de vision ne s'est produite à la fin de la période d'observation et 1 indique qu'une perte de vision s'est produite.
\end{itemize}

Ces données proviennent d'une étude d'analyse de la survie de patients atteints de rétinopathie diabétique à haut risque, conçue pour évaluer l'efficacité du traitement au laser dans le ralentissement de la progression de la cécité.

\section{b}
Nous avons obtenu un graphique montrant la probabilité de survie pour l'ensemble de l'échantillon depuis le début du traitement jusqu'à la cécité (définie comme une baisse de l'acuité visuelle à 5/200). La courbe commence à 1 (soit une probabilité de survie de 100 %) et diminue progressivement avec le temps, ce qui suggère que de plus en plus de patients atteignent la définition de la cécité au fil du temps. Sur l'axe des abscisses, le temps est indiqué en jours, tandis que l'axe des ordonnées montre la proportion de patients qui n'ont pas connu d'épisode de cécité à un moment donné.

\begin{lstlisting}
#Kaplan-Meier 
result.km <- survfit(Surv(time, status) ~ 1, conf.type="log-log")
plot(result.km, xlab = "Jours", ylab = "Probabilité de Survie", main = "Kaplan-Meier")

\end{lstlisting}

\section{c}


\end{CJK*}
\end{document}
