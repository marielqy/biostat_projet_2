\documentclass[../main.tex]{subfiles}

\begin{document}
\begin{CJK*}{UTF8}{gbsn}
\section*{Exercice 1}
Décrivez brièvement les variables du jeu de données 
\texttt{diabetic} dans la bibliothèque \texttt{survival} en R.
Quelle était une question de recherche menant à cette collecte de données?
Produisez un graphique de Kaplan-Meier utilisant l'échantillon entier,
le groupe de \texttt{trt = 0} et le groupe de \texttt{trt = 1} 
pour le temps jusqu'à devenir aveugle.
Pour le groupe \texttt{trt = 0}, trouvez le temps médian de survie et
construisez un intervalle de confiance à $95\%$ pour le temps médian de survie.
Faites un test de Log-Rang pour comparer les deux groupes.

Finalement, on se demande si d'autres variables
dans le jeu de données pourraient être des facteurs confondants, et si l'on devrait stratifier le
test du Log-Rang sur l'une de ces variables. À partir de statistiques descriptives, de graphiques
et/ou d'arguments adaptés au contexte de l'étude, discuter de laquelle des variables du jeu de
données risque d'agir comme facteur confondant, et reproduire le test du Log-Rank stratifié pour cette variable.

\paragraph{Solution}

à faire

\begin{lstlisting}
quelque listing

\end{lstlisting}

\end{CJK*}
\end{document}
