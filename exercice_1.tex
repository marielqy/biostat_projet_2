\documentclass[../main.tex]{subfiles}

\begin{document}
\begin{CJK*}{UTF8}{gbsn}
\section*{Exercice 1}
Décrivez brièvement les variables du jeu de données 
\texttt{diabetic} dans la bibliothèque \texttt{survival} en R.
Quelle était une question de recherche menant à cette collecte de données?
Produisez un graphique de Kaplan-Meier utilisant l'échantillon entier,
le groupe de \texttt{trt = 0} et le groupe de \texttt{trt = 1} 
pour le temps jusqu'à devenir aveugle.
Pour le groupe \texttt{trt = 0}, trouvez le temps médian de survie et
construisez un intervalle de confiance à $95\%$ pour le temps médian de survie.
Faites un test de Log-Rang pour comparer les deux groupes.

Finalement, on se demande si d'autres variables
dans le jeu de données pourraient être des facteurs confondants, et si l'on devrait stratifier le
test du Log-Rang sur l'une de ces variables. À partir de statistiques descriptives, de graphiques
et/ou d'arguments adaptés au contexte de l'étude, discuter de laquelle des variables du jeu de
données risque d'agir comme facteur confondant, et reproduire le test du Log-Rank stratifié pour cette variable.

\paragraph{Solution}
\section{a}
\begin{itemize}
  \item \textbf{ID} : Il est utilisé pour distinguer chaque participant dans l'ensemble de données.
  \item \textbf{laser} : Il s'agit du type de traitement laser reçu. 1=xenon, 2=argon
  \item \textbf{age} : Il s'agit de l'âge auquel le diabète a été diagnostiqué chez le patient.
  \item \textbf{eye} : Il s'agit d'un facteur avec des niveaux de gauche et de droit.
  \item \textbf{trt} : Il s'agit du groupe de traitement. 0=no treatment, 1=laser
  \item \textbf{risk} : Il s'agit d'une variable quantitative d'évaluation du risque utilisée pour classer les participants dans différents groupes de risque (les valeurs varient de 6 à 12).
  \item \textbf{time} : Il s'agit de la date de l'événement ou de la dernière visite de suivi.
  \item \textbf{status} : Il s'agit d'une variable binaire utilisée pour indiquer si une perte de vision s'est produite au cours de la période d'étude, où 0 indique qu'aucune perte de vision ne s'est produite à la fin de la période d'observation et 1 indique qu'une perte de vision s'est produite.
\end{itemize}

Ces données proviennent d'une étude d'analyse de la survie de patients atteints de rétinopathie diabétique à haut risque, conçue pour évaluer l'efficacité du traitement au laser dans le ralentissement de la progression de la cécité.

\section{b}
Nous avons obtenu un graphique montrant la probabilité de survie pour l'ensemble de l'échantillon depuis le début du traitement jusqu'à la cécité (définie comme une baisse de l'acuité visuelle à 5/200). La courbe commence à 1 (soit une probabilité de survie de 100 %) et diminue progressivement avec le temps, ce qui suggère que de plus en plus de patients atteignent la définition de la cécité au fil du temps. Sur l'axe des abscisses, le temps est indiqué en jours, tandis que l'axe des ordonnées montre la proportion de patients qui n'ont pas connu d'épisode de cécité à un moment donné.

\begin{lstlisting}
#Kaplan-Meier 
result.km <- survfit(Surv(time, status) ~ 1, conf.type="log-log")
plot(result.km, xlab = "Jours", ylab = "Probabilité de Survie", main = "Kaplan-Meier")

\end{lstlisting}

\section{c}
Le graphique généré par le code r comporte trois courbes qui représentent la probabilité de survie de l'échantillon entier (représenté en noir), du groupe non traité (représenté en rouge, trt = 0) et du groupe traité (représenté en bleu, trt = 1). Les graphiques montrent que la courbe de survie du groupe traité est plus élevée que celle du groupe non traité à la plupart des moments, ce qui suggère que le traitement peut aider à retarder l'apparition de la cécité.

\begin{lstlisting}
result.kmtrt0 <- survfit(Surv(time[trt == 0], status[trt == 0]) ~ 1, conf.type="log-log")
par(new=TRUE)
plot(result.kmtrt0, xlab = "Jours", ylab = "Probabilité de Survie", main = "Kaplan-Meier", col = "red")
result.kmtrt1 <- survfit(Surv(time[trt == 1], status[trt == 1]) ~ 1, conf.type="log-log")
par(new=TRUE)
plot(result.kmtrt1, xlab = "Jours", ylab = "Probabilité de Survie", main = "Kaplan-Meier", col = "blue")
legend("topright", legend=c("totale","trt = 0", "trt = 1"), col=c("black","red", "blue"), lty=1:5, cex=0.8)
#abline ( v = 43.7 , col = 'red' , lty =2)
\end{lstlisting}

\section{d}

Le but de ce problème est de se concentrer sur le groupe de patients qui n'ont pas reçu de traitement au laser (i.e., trt = 0) et de trouver le temps de survie médian dans ce sous-ensemble, qui est défini ici comme le temps écoulé entre le début du traitement et la cécité. Nous devons ensuite calculer un intervalle de confiance à 95\% pour ce temps de survie médian, et nous avons choisi d'utiliser la méthode "log-log" fournie par Barker (2009).

Le code R crée d'abord un sous-ensemble des données \texttt{subsetdata} en filtrant les patients avec \texttt{trt == 0}de l'ensemble de données \texttt{diabetic} à l'aide de la fonction subset. Ensuite, nous avons utilisé la fonction \texttt{survfit} pour estimer la durée de survie médiane de ce sous-ensemble et l'intervalle de confiance à 95\% correspondant, où nous avons choisi le type d'intervalle de confiance "log-log".

Selon les résultats obtenus :
\begin{enumerate}
  \item Chez les patients n'ayant pas reçu de traitement (\texttt{trt = 0}), la durée médiane de survie est de 43,7 jours.
  \item L'intervalle de confiance à 95\% a une limite inférieure de 31,6 jours et une limite supérieure de 59,8 jours.
\end{enumerate}

\begin{lstlisting}
subset_data <- subset(diabetic,trt == 0)
fit <- survfit(Surv(time,status) ~ 1,data=subset_data, conf.type="log-log")
result.km<-fit
print(result.km)
Call: survfit(formula = Surv(time, status) ~ 1, data = subset_data, 
    conf.type = "log-log")

       n events median 0.95LCL 0.95UCL
[1,] 197    101   43.7    31.6    59.8
\end{lstlisting}

\end{CJK*}
\end{document}
