\documentclass[../main.tex]{subfiles}

\begin{document}
\begin{CJK*}{UTF8}{gbsn}
\section*{Exercice 1e}
Soit $\{Y_i\}_{i=1}^m$, $\{a_i\}_{i=1}^m$, $\{b_i\}_{i=1}^m$, $\{\epsilon_i\}_{i=1}^m$
trois suites de variables aléatoires à valeur $\mathbb{R}^5$, $\mathbb{R}$, $\mathbb{R}$ et $\mathbb{R}^5$
respectivement tout définies sur une espace de probabilité commune $(\Omega, F, P)$. 
On suppose que :

\begin{enumerate}
    \item Les quatres $\sigma$-algébres générées par les quatre suites sont indépendantes.
    \item Il y existe une matrice $X$ de taille $5 \times 5$. Pour tout $j \in \{1, \dots, 5\}$, on a, presque partout :
    
    \begin{equation*}
        Y_{ij} = a_i + b_i X_{ij} + \beta X_{ij} + \epsilon_{ij}
    \end{equation*}
    \item Il existe constantes $\tau > 0$, $\psi > 0$, $\sigma > 0$ et $\rho \in (0,1)$ tels que, 
    pour tout $i \in \{1, \cdots, m\}$,
    $a_i$ est la loi normale avec moyen $4$ et variance $\tau^2$, 
    $b_i$ est la loi normale avec moyen $0$ et variance $\psi^2$ et:

    \begin{equation*}
        \text{Cov}(\epsilon_{i}) = \sigma^2
        \begin{bmatrix}
            1 & \rho & \rho^2 & \rho^3 & \rho^4 \\
            \rho & 1 & \rho & \rho^2 & \rho^3 \\
            \rho^2 & \rho & 1 & \rho & \rho^2 \\
            \rho^3 & \rho^2 & \rho & 1 & \rho \\
            \rho^4 & \rho^3 & \rho^2 & \rho & 1
        \end{bmatrix}
    \end{equation*}
\end{enumerate}

Séparez et présentez le formule de $Y_{ij}$ comme deux composantes: une composante de régression
linéaire et une erreur. Calculer $\text{Cov}(Y_i)$ pour chaque $i \in \{1, \dots, m\}$.

\paragraph{Solution} 

Comme la partie de régression linéaire contient les termes déterministes ou systématiques du modèle,
le modèle peut être séparé comme, pour tout $i \in \{1, \dots, m\}$ et $j \in \{1, \dots, 5\}$:

\[
Y_{ij} = a_i + (b_i + \beta) X_{ij} + W_{ij}
\]

Ici \(a_i + (b_i + \beta) X_{ij}\) est la partie de régression linéaire et \(W_{ij}\) est la partie d'erreur.
Ensuite, comme les quatre suites sont indépendants, 
on obtient, pour chaque $i \in \{1, \dots, m\}$ et $j \in \{1, \dots, 5\}$, que :

\begin{equation*}
\text{Var}(Y_{ij}) = \text{Var}(a_i) + \text{Var}(b_i X_{ij}) + \text{Var}(\epsilon_{ij})
\end{equation*}

En utilisant les normalités assumées :

\begin{equation*}
\text{Var}(Y_{ij}) = \tau^2 + \psi^2 X_{ij}^2 + \sigma^2
\end{equation*}

Si $j \neq k$, on a :

\begin{equation*}
\text{Cov}(Y_{ij}, Y_{ik}) = \text{Cov}(a_i + b_i X_{ij} + \epsilon_{ij}, a_i + b_i X_{ik} + \epsilon_{ik})
=  \tau^2 + \psi^2 X_{ij} X_{ik} + \sigma^2 \rho^{|j-k|}
\end{equation*}

Le calcul est complet. ////

\end{CJK*}
\end{document}
